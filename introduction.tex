%\input{preamble}

%\begin{document}
\section{Introduction}

Container shipping is a central element in the clockwork of global trade. In fact it is believed to be more important for globalization than freer markets \cite{EC13}. A container liner shipping company operates a set of container vessels. The vessels sail on closed loop services with fixed  schedules that connect major trade regions like Asia and Europe. Liner shipping business is focused on utilizing the cargo capacity in the service network. Unused capacity is a loss and the competition is fierce with a profit margin of just a few percent.  

For that reason, it is central for liner shipping companies to be able to estimate the residual capacity of a container vessel. This is much harder than in the airline industry, where the number of empty seats in an airplane usually equals its free passenger capacity. For a container vessel, on the other hand, it can be hard to determine whether an empty container slot can be filled. The weight capacity of the container stack of the slot or its lashing gear may be exhausted. It may be impossible to place the type of container we want in the slot due stacking rules such as no 20' long on top of 40' long containers and dangerous goods separation rules. Placing a container may break stress limits such as torsion moment and shear forces or cause the vessel to become unstable. The container we want to place may also block for containers below it in the stack that must be discharged in next port or it may cause the crane assigned to the stack to have too many moves to carry out. 

Often it is only the stowage planning experts of a service that can determine the residual capacity of a vessel accurately and even they may sometimes have to manually construct a stowage condition of the vessel to express this capacity in terms of a trade-off between the different container types to load.

Many other functions in a liner shipping company, however, depend on accurate capacity estimates. These include: {\em uptake management} that control the sale of cargo bookings to fill the vessels with profitable cargo; {\em capacity management} that route cargo through the service network; {\em network management} that makes changes to the service network; and {\em fleet management} that charters and buys vessels for the service network and reposition vessels between closing and opening services.  

Decision makers in these functions seldom have deep technical insight and traditionally regard vessel capacity in terms of maximum volume, maximum weight, and maximum number of reefer containers (refrigerated containers). They need to consult the stowage division to determine vessel capacity accurately. This is resource and time consuming and often skipped in practice.      

Previous work on stowage planning optimization (e.g., cite cite) has contributed frameworks for automated stowage planning. Recently, mixed integer programming (MIP) stowage planning models have been shown to scale to large container vessels (cite pacino, mm). These models can be used to compute the


 Master planning is an abstraction of stowage planning, where    
containers to load are grouped according to type and vessels slots are grouped according to position on the vessel. 


%- previous work on stowage planning: not practical too large and detailed linear stowage models quite accurate
%- Follow previous work on applying FME to simplify system
%   - introduce a novel framework
%   - in addition to previous work takes advantage of block angular structure
%   - results framework works on container vessels
%   - probably works in a wide range of problems "block angular"
%


%\end{document}
