%\documentclass[a4paper,twoside,10pt]{article}
\interfootnotelinepenalty=10000
\usepackage[USenglish]{babel} %francais, polish, spanish, ...
\usepackage[T1]{fontenc}
%\usepackage[ansinew]{inputenc}
\usepackage{color}
\usepackage{mathtools}
%\usepackage{hyperref}
\usepackage{subfig}
\usepackage{multirow, booktabs}
\usepackage{hyperref}


\usepackage{lmodern} %Type1-font for non-english texts and characters
\usepackage{algorithm}
\usepackage[noend]{algpseudocode}
\usepackage{mnsymbol}

%% Packages for Graphics & Figures %%%%%%%%%%%%%%%%%%%%%%%%%%
\usepackage{graphicx} %%For loading graphic files
\usepackage{amsmath}
\usepackage{amsthm} 
\usepackage{thmtools}
\usepackage{amsfonts}
\usepackage[all,cmtip]{xy}
\usepackage{tikz}

\usepackage{TechFront}
%\declaretheorem{Lemma}
%\declaretheorem{prop}

\newcommand{\lre}{\color{red}{\{}}

%\DeclareMathOperator{\sign}{sgn}
%\DeclareMathOperator{\coef}{coef}
%\DeclareMathOperator{\var}{var}
%\DeclareMathOperator{\eqs}{eqs}
%\DeclareMathOperator{\feas}{feas}
%\DeclareMathOperator{\UB}{UB}
%\DeclareMathOperator{\lb}{lb}
%\DeclareMathOperator{\FMcomb}{FM-comb}
%\DeclareMathOperator{\Gcomb}{Gauss-comb}
%\DeclareMathOperator{\proj}{proj}
%\DeclareMathOperator{\Pos}{Pos}
%\DeclareMathOperator{\Neg}{Neg}
%\DeclareMathOperator{\rhs}{rhs}
\newcommand{\sign}{\mathit{sgn}}
\newcommand{\coef}{\mathit{co}}
\newcommand{\var}{\mathit{var}}
\newcommand{\VAR}{\mathit{VAR}}
\newcommand{\eqs}{\mathit{eqs}}
\newcommand{\feas}{\mathit{feas}}
\newcommand{\UB}{\mathit{UB}}
\newcommand{\UBc}{\mathit{UBineq}}
\newcommand{\lb}{\mathit{lb}}
\newcommand{\lbc}{\mathit{lbineq}}
\newcommand{\FMcomb}{\mathit{FM}}
\newcommand{\Gcomb}{\mathit{GA}}
\newcommand{\proj}{\mathit{proj}}
\newcommand{\Pos}{\mathit{Pos}}
\newcommand{\Neg}{\mathit{Neg}}
\newcommand{\rhs}{\mathit{rhs}}
\newcommand{\bounds}{\mathit{bounds}}
\newcommand{\ie}{\mathcal{IE}}
\newcommand{\xx}{\mathcal{X}}
\newcommand{\vea}{\mathbf{co}}
\newcommand{\ttt}{\texttt{t}}
\newcommand{\trt}[1]{\texttt{#1}}
\newcommand{\mi}{\mathit}

\newcommand{\false}{\texttt{false}}
\newcommand{\true}{\texttt{true}}
\newcommand{\nul}{\texttt{null}}
\newcommand{\ve}{\mathbf}
%\newcommand\lhs[1]{\text{lhs}(#1)}
%\newcommand\rhs[1]{\text{rhs}(#1)}
%\newcommand\coef[1]{\text{coef}(#1)}
%\newcommand\LB[1]{\text{LB}_{#1}}
%\newcommand\UB[1]{\text{UB}_{#1}}
\newcommand{\lig}[4]{\ve{#1}\cdot\ve{#2}#3#4}
\newcommand\red[1]{\textcolor{red}{#1}}
\newcommand\blue[1]{\textcolor{blue}{#1}}
\newcommand{\set}[2]{\{\;{#1}\;|\;{#2}\;\}}
\newcommand{\odef}{\overset{\text{def.}}=}
\newcommand{\mc}{\mathcal}
\algdef{SE}[DOWHILE]{Do}{doWhile}{\algorithmicdo}[1]{\algorithmicwhile\ #1}%
\newcommand{\argmin}{\operatornamewithlimits{argmin}}
\newcommand{\StateInd}{\State\hspace{\algorithmicindent}}
\newcommand{\pr}{\mathit{PR}}
\newcommand{\prs}{\mathit{PRS}}
\newcommand{\ens}{\Leftrightarrow}

%\algdef{SE}[DOPAR]{DoPar}{doParWhile}{\algorithmicdo\textbf{ in parallel for\ }}[1]{\algorithmicwhile\ #1}%
\algdef{SE}[DOPAR]{DoPar}{doParUntil}{\algorithmicdo\textbf{ in parallel for\ }}[1]{\algorithmicuntil\ #1}%

\algdef{SE}[SUBALG]{Indent}{EndIndent}{}{\algorithmicend\ }%
\algtext*{Indent}
\algtext*{EndIndent}

\newtheorem{prop}{Proposition}
\newtheorem{lemma}{Lemma}
\newtheorem{cor}{Corollary}

\newcounter{para}
%\newcommand\mypara[1]{\par\refstepcounter{para}\textbf{\thep‌​ara\space#1\space}}
\newcommand\mypara[1]{\newline\par\refstepcounter{para}\textbf{\thepara}\space \textbf{#1} \space}
%\newcommand\mypara{\par\refstepcounter{para}\thepara\space}
%\usepackage[thmmarks,...]{ntheorem}
\newcommand{\Sec}{F}
\newcommand{\Ca}{\mi{Cap}}
\newcommand{\Vol}{\mi{V}}
\newcommand{\Weight}{\mi{W}}
\newcommand{\weight}{\mi{w}}
\newcommand{\BonjeanStations}{\mi{BS}}
\newcommand{\bonjean}{bf}
\newcommand{\Bonj}{B}
\newcommand{\shear}{\mi{sf}}
\newcommand{\Prop}{P}

\theoremstyle{definition}
%\newtheorem{example}{Example}[section]
\newtheorem*{theorem}{Theorem}

\theoremstyle{definition}
\newtheorem{examplex}{Example}[section]
\newenvironment{example}
  {\pushQED{\qed}\renewcommand{\qedsymbol}{$\triangle$}\examplex}
  {\popQED\endexamplex}
	
%\newtheoremstyle{named}{}{}{\itshape}{}{\bfseries}{.}{.5em}{\thmnote{#3}}
%\theoremstyle{named}
%\newtheorem*{namedtheorem}{Theorem}


%\begin{document}
\section{Introduction}
\paragraph{Liner shipping}
Container shipping is a central element in the clockwork of global trade. In fact, it is believed to be more important for globalization than freer markets \cite{EC13}. A container liner shipping company operates a set of container vessels which sail on closed loop services with fixed schedules. These services connect major trade regions like Asia and Europe, and the liner shipping business is focused on utilizing the cargo capacity in their service network. Unused capacity constitute a loss and the competition is fierce with a profit margin of just a few percent.  

To avoid unused capacity, it is central for a liner shipping company to be able to estimate the residual capacity of a partially loaded container vessel. However, unlike e.g. an airplane, where the number of empty seats usually equals its free passenger capacity, it is hard to determine whether an empty container slot in a container vessel can be filled. % since various stowage rules interact to restrict the permitted cargo.  
For example, the weight capacity of the container stack of the slot or the stack's lashing gear may already be exhausted. 
Similarly, placing a container in the slot may break stress limits such as torsion moment and shear forces or cause the vessel to become unstable.
Further, placing a container in the slot on top of containers that must be discharged in the next port of call may obstruct the discharge of these or assign too many moves to the crane working on the stack in the port of discharge. 
%
Additionally, it may be impossible to place a specific \emph{type} of container in a slot. This may be due to rules regarding separation of dangerous goods, or because of stacking rules that e.g. disallow 20' long containers to be stacked on top of 40' long containers. 

Often it is only the stowage planning experts of a service that can determine the free capacity of a vessel accurately, and even they may sometimes have to manually construct a stowage condition of the vessel to be able to express the residual capacity in terms of a trade-off between the different container types to load.

\paragraph{Usage of capacity models}
Accurate capacity estimates are, however, not only important in stowage 
for securing good utilization of the vessels,
it is also paramount for many other functions in a liner shipping company. These include: {\em uptake management} that control the sale of cargo bookings such as to fill the vessels with profitable cargo; {\em capacity management} that route cargo through the service network; {\em network management} that makes changes to the service network; and {\em fleet management} that charters and buys vessels for the service network and reposition vessels between closing and opening services.  

Decision makers in these functions seldom have deep stowage insight and traditionally regard vessel capacity in terms of maximum volume, maximum weight, and maximum number of reefer containers (refrigerated containers). In order to determine vessel capacity accurately they need to consult the stowage team, and since this is resource and time consuming, they often act on insufficient information leading to sub-optimal decisions in these high functions.  

\paragraph{Stowage models as capacity models}
Previous work on stowage planning optimization (e.g., \cite{roach00,kimkang02,ambrosino04,low09,delgado09,pacino12}) has contributed frameworks for automated stowage planning, 
%shouldn't it be mentioned that is is within linear programming?
and recently, mixed integer programming (MIP) stowage planning models were shown to scale to large container vessels (\cite{pacino11,AlbertosThesis}). Since these models define the set of legal stowage conditions, they can be used to compute the residual capacity of a container vessel as a result of the interaction between the different types of containers to load. In other words, they embed an accurate capacity model. 

However, the use of these stowage models as capacity models is limited due to their size. Since a stowage model not only models the capacity of a vessel as a function of the different container types to load but also describes the position of the containers on the vessel, it may contain thousands of constraints and variables and take long time to solve. This is a problem if the stowage model is to be used as a capacity model %. The reason is that 
since optimization in uptake-, capacity-, fleet-, and network management often requires several hundred capacity models to be solved simultaneously. For example, the task in uptake management of maximizing uptake over a set of voyages, requires us to associate a capacity model with each individual leg of the voyages and optimize the resulting model.  

%%%%%%%%%%%%%%%%%%%%%%%%%%%%%%%%%%%%%%%%%%%%%%%%%%%%%%%%%%%%%%%%%%%
\paragraph{Our contribution}
In this article, we present a method to calculate reasonable sized capacity models automatically from the linear programming models contributed by previous research on stowage optimization. The basic idea is to abstract away the information about the containers' positions so that the resulting model only depends on variables denoting the total amount of each possible type of containers, making it a capacity model. {The derived capacity models are valid in the sense that each feasible solution in the capacity model is associated with a feasible solution in the stowage model it is derived from, even though the capacity model has no information about the place at which the containers are stowed.}

Inspired by previous work in constraint programming (\cite{lassez90}) and software verification (\cite{benoy05}), we do this by projecting container position variables out of the mentioned stowage models using a framework based on Fourier-Motzkin Elimination (FME). 
Similar to other recent FME frameworks (e.g., \cite{simon05,lukatskii08,shapot12}), our framework includes additional features such as using equalities for substitutions (Gauss-elimination), removing syntactically redundant constraints, a complete removal of redundant constraint after each variable elimination, and a method for coarsening the boundary of the projection. 

Our main contribution is a novel decomposition method that takes advantage of the block-angular structure of stowage models to significantly speed-up the projection of variables in these models. Additionally, the complete redundancy is parallellized, % sound, concurrent implementation of the redundancy removal.
and the framework includes preprocessing of the constraint system including removal of less strict inequalities.

\paragraph{Results}
A drawback of FME is that for each variable projected, the number of constraints in the resulting system can increase quadratically, leading to a theoretical double exponential complexity in the number of variables to project. However, our experimental evaluation of our FME framework on stowage models shows that almost all constraints added by FME are redundant or negligible, and can accordingly be removed. 

Our experiments likewise show that for complex systems, the decomposition method reduce the size of the intermediary systems produces by FME, causing less time to be needed for removing redundant constraints, which again speeds up the projection process significantly. This speed-up is essential for the projection of the more complex models, since otherwise the projection times out after 65 hours having only projected a fraction of the required variables.

The results likewise show that the number of constraints and non-zeros in the resulting capacity models typically are reduced by an order of magnitude compared to their corresponding stowage models. Moreover, in our experiments, the resulting model can be solved 20-130 times faster than their stowage models with only a negligible loss in accuracy. Although it can take several hours of CPU time to derive a capacity model, this only has to be done one time, and in this way, our FME framework performs an off-line computation that factors out time when applying the capacity models as sub-models for uptake-, capacity-, fleet-, and network optimization.

\paragraph{Multi-commodity flow}
It is not only for stowage models within the domain of liner shipping that it is meaningful to eliminate variables irrelevant for an analysis of a model at a ``higher level''. Consider e.g. a multi-commodity flow graph, where commodities flow through the graph from sources to sinks, and each edge have limits for each commodity as well as a joint limit. In a scenario where the supply and demand of each commodity is not fixed it could be of interest to know how the amount of each commodity at the sinks depends on the amount of the commodities at the sources. For this purpose, the amount of each commodity that flows at each edge of the graph is irrelevant, and to get a description of the direct relationship between the inflow-variables and the outflow-variables, the other variables should be eliminated from the system describing the multi-commodity flow problem.

Since the block-angular structure is frequent in LP models, our decomposed FME framework applies to other problems as well. To that end, as an example we have examined a multi-commodity problem as described above and found a speed-up of the projection process similar to the one seen for capacity models.

\paragraph{Structure of article}
This article is organized as follows. First we introduce the required definitions and notation. This is followed by a section outlining the methods used in our FME framework. Subsequently we detail how block-angular problems are decomposed, after which we present the vessel models (stowage models) that are projected. Finally follows our experimental results before we conclude.

%\bibliography{bibfile}{}
%\bibliographystyle{alpha}

%\end{document}
