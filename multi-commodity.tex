\input{preamble}
\begin{document}
\subsection*{Use in multi-commodity flow graphs}
Elimination of variables irrelevant for a more general/``higher level'' question is not only limited to the considered task of obtaining a cargo model from a stowage model. 
As another example, consider eg. a multi-commodity flow graph $G=(V,E)$, where commodities $c_1$ to $c_k$ flow through the graph from sources ($S\subseteq V$) to sinks ($T\subseteq V$), and each edge have limits for each commodity as well as a joint limit. 
For a number of applications it would be of interest to know how much can flow from the source-nodes to the sinks, or more specifically, how the amount/number of each commodity at the sinks ($x_{t, c_i}$ for all $t\in T$ and $1\leq i\leq k$) depends on the amount/number of the commodities at the sources ($x_{s, c_i}$ for all $s\in S$ and $1\leq i\leq k$)\footnote{Notice that in this scenario, the demand of each commodity is not fixed/given.}. For this purpose, we are uninterested in the amount/number of each commodity that flows at each edge of the graph, and the variables denoting this could/would/should/... therefore be eliminated from the original model/system describing the multi-commodity flow problem. 
		
A multi-commodity flow problem is naturally block-structured, though, there are usually many global constraints -- corresponding to the common upper limits for each edge. Instead of using these blocks to decompose the system, it is also possible to divide the graph into smaller subgraphs. We then treat the ingoing edges in a subgraph $G'$ as sources ($S_{G'}$) and the outgoing edges as sinks ($T_{G'}$), and eliminate all variables but $\set{x_{v,c_i}}{v\in S_{G'}\cup T_{G'}}$ from the inequality systems describing the flow problem in $G'$. Afterwards, these  subgraphs are successively combined into larger graphs $\tilde{G}$ from which all but $\set{x_{v,c_i}}{v\in S_{\tilde G}\cup T_{\tilde G}}$ are eliminated. 

\paragraph{[Results]}
We have generated problems [inspired by problems that can be found in...]. ... Simple: [Drawing] Two (or three) nodes at each level, levels following each other; from each level, there are edges to all other nodes at the next level. Sources are composed of the nodes at the first level, while the nodes at the last level constitutes the sinks. The capacity for each commodity $c_i$ and edge $e$  0 with a probability of .. and otherwise randomly generated from a [normal distribution between ... and ...], while the common capacity of the edge e is 0 with probability ... and otherwise a random number generated from the [normal distribution between ... and [something with the sum]].
\end{document}