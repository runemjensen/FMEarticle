\input{preamble}
\algdef{SE}[DOWHILE]{Do}{doWhile}{\algorithmicdo}[1]{\algorithmicwhile\ #1}%

\begin{document}
\subsection*{Overall procedure}
Combining all of the above mentioned step, the overall projection procedure can thus be described with the following pseudocode:
%
\begin{algorithmic}
\Function{Project}{$S$,$Y$}
	\State $(S,Y)\gets \Call{Preprocess}{S,Y}$
	\State $(S,Y)\gets\Call{Gauss-Elim}{S,Y}$
	\While{$Y\neq\emptyset$}
		\State $(S, Y, New )\gets\Call{FME-SingleVar}{S,Y}$
		\State $S\gets\Call{RemoveRedundancy}{S,New}$
	\EndWhile
	\State \Return $S$
\EndFunction
\end{algorithmic}

The sub-algorithms used are:
\begin{algorithmic}
\Function{Preprocess}{$S, Y$}
	\Do
		\State $(S,Y)\gets$ remove unused variables (in $Y$) and and empty (in)equalities
		\State $(S,Y)\gets$ do FME on variables only occurring with only one sign
		\State Update bounds for all variables and the left-hand-side of (in)equalities
		\State $S\gets$ remove syntactically redundant inequalities
	\doWhile{A variable or (in)equality is removed, a variable is replaced with a value, or a bound is made stricter}
\State $S \gets$ add inequalities expressing the non-trivial bounds for the variables 
\State \Return $(S, Y)$
\EndFunction
\end{algorithmic}

\begin{algorithmic}
\Function{Gauss-Elim}{$S$, $Y$}
\While{$Y$ contains a variable used in an equality in $S$}
	\State $Y'\gets$ the variables in $Y$ that are used in any equality in $S$
%	\State \color{red}{or: $Y'\gets \set{x\in Y}{x\in \mi{var}(\{e\})\text{ for any equality } e\in S}$}
	\State \color{black}{$x\gets$ the variable in $Y'$ used by the fewest (in)equalities in $S$}
%	\State \color{red}{or: $x\gets$ the variable in $Y'$ that minimizes $|\set{c\in S}{x\in \mi{var}(\{c\})}|$}
	\State \color{black}{$e\gets$ the equality in $S$ using $x$ that uses the fewest variables} %minimizes $|\mi{var}(\{e\})|$}
	\State Remove $e$ from $S$ and $x$ from $Y$ %$S\gets S\setminus \{e\}$, $Y\gets Y\setminus\{x\}$
	\State Isolate $x$ in $e$ and substitute in all (in)equalities in $S$
\EndWhile
\State \Return $(S,Y)$
\EndFunction
\end{algorithmic}

\begin{algorithmic}
\Function{FME-SingleVar}{$S,Y$}
	\State $x\gets$ the variable in $Y$ that minimizes $|\Pos_S(x)\cdot \Neg_S(x) - \Pos_S(x)-\Neg_S(x)|$
	\State $S'\gets \set{i_{p,n,x}}{p\in \Pos_S(x), n\in  \Neg_S(x)}$
	\State \Return $(\mi{Zero}_S(x)\cup S', Y\setminus\{x\}, S')$
\EndFunction
\end{algorithmic}

\begin{algorithmic}
\Function{RemoveRedundancy}{$S$, $N$}
	\State Examine each inequalities in $N$ in parallel:
	\State $\quad$ $R\gets$ the inequalities that are strictly redundant w.r.t. $S$
	\State $\quad$ $A\gets$ the inequalities that are almost redundant w.r.t. $S$
	\State $S\gets S\setminus R$
	\State Examine all inequality in $A$, sequentially:
	\State $\quad$ $A'\gets$ the inequalities that are almost redundant w.r.t. $S$
\EndFunction
\State \Return $S\setminus (R\cup A')$
\end{algorithmic}	

N�h ja, og s� er der lige list ``clean-up'' ind imellem. Tror ikke, jeg har brug for at beskrive det med $\kappa-values$, da det ikke er aktuelt i de konkrete projektioner...(?)
\\\\
Each time we further decompose a system, we use a partitioning of the subsystems $S^l_1, \ldots, S^l_{k_l}$ to create $k_{l+1}$ new subsystems $S^{l+1}_1,\ldots, S^{l+1}_{k_{l+1}}$. This creates a new ``level'' of subsystems and results in a tree structure of smaller inequality systems, where each node is associated with a subsystem and a set of variables that should be eliminated from the system. \red{The leafs are associated with the systems $S^0_1,\ldots, S^0_k$ and the variable sets $Y\cap \mi{VAR}(S_1),\ldots, Y\cap \mi{VAR}(S_k)$, respectively, while the root is associated with the the pair $(S^K_\trt{g}, \set{z^K_{c,j}}{(c,j)\in S_\trt{g}\times\{1,\ldots,k_K\}}\cup Y')$ for some $K$, where $Y'=Y\setminus \mi{VAR}(S_1)\cup\ldots \cup \mi{VAR}(S_k)$.  
For $0<l\leq K$, a node $n$ associated with the system $S^l_i=\set{\mi{Def}(z^l_{c,i})}{c\in S_\trt{g}}$
is associated with the variable set 
$\set{z^{l-1}_{c,j}}{(c,j)\in S_\trt{g}\times \{1,\ldots, k_{l-1}\}}\cap \VAR(S^l_i)$.}

To project $S$ w.r.t. $Y$ we therefore create the tree $T$ as decribed above \red{and ... recursively..}
That is, we call \Call{ProjectNode}{root of $T$}, where

\begin{algorithmic}
\Function{ProjectNode}{Node $n$} 
	\State $(S,Y)\gets$ the system and variable set associated with $n$
	\If{$n$ is a leaf}
		\State \Return \Call{FM-ProjectionFramework}{$S$, $Y$}
	\Else
		\ForAll{children $m$ of $n$}
			\State $S \gets S\cup \Call{ProjectNode}{m}$
		\EndFor
		\State \Return $\Call{FM-ProjectionFramework}{S, Y}$
	\EndIf
\EndFunction
\end{algorithmic}

Parallelization
\begin{algorithmic}
\Function{Manager}{$T$ tree structure of $S$}
	\State Initialize the count of all nodes to $0$ and all projections to $\emptyset$
	\State Create $w$ workers, initially idle
	\State Initialize a queue $Q$ with all leaves in $T$
	\While{the projection of $T$ equals $\emptyset$}
		\If{$Q$ is non-empty}
			\State As soon as a worker $w$ is idle
			\State Remove first node $n$ from $Q$
			\State Give $w$ $n$ as input
		\EndIf
	\EndWhile
\EndFunction
\State \Return the projection of the root of $T$
\Statex
\Function{Worker}{node $n$}
\State the projection of $n\gets$ \Call{Project}{System $S$ of $n$, elimination variables of $n$}
\State $p\gets$ parent of $n$
\State Increase count of $p$
\If{the count of $p$ equals the number of $p$'s children}
	\State add p to $Q$
\EndIf
\State \Return to idle state
\EndFunction
\end{algorithmic}	
 

\end{document}