\subsection{Preprocessing}
Prior to projecting the system we perform some simple preprocessing steps in order to have a smaller system as the starting point. The steps (which can be found in e.g. \cite{brearley75,andersen95,maros})
are then applied repeatedly in a cycle, as long as ``something happens'' in a cycle, that is, a constraint or variable in $Y$ is removed, a variable is substituted with a value, or a bound of a variable or an (in)equality's left-hand-side is updated. The steps are implemented assuming that the system is feasible.

When performing the preprocessing steps, for each variable $x$ we maintain an upper and lower bound, $ub_x$ and $lb_x$, which initially is $+\infty$ and $-\infty$, respectively. When the bounds are finite, they are treated as inequalities in the system. For each constraint $c$, we likewise maintain an upper and lower bound, $ub_c$ and $lb_c$ for its left-hand-side {based on the bounds of the used variables}. That is, for the inequality $c$ with left-hand-side $\ve{a}\cdot\ve{x}$, we define $ub_c = \smashoperator[r]{\sum_{x. a_x>0}}a_x\cdot ub_x + \smashoperator[r]{\sum_{x.a_x<0}}a_x\cdot lb_x$ and $lb_c = \smashoperator[r]{\sum_{x. a_x>0}}a_x\cdot lb_x + \smashoperator[r]{\sum_{x.a_x<0}}a_x\cdot ub_x$. 
This links the bounds of variables and constraints, and thus, %links The bounds of variables and inequalities are linked, since for the inequality $c$ with left-hand-side $\ve{a}\cdot\ve{x}$ we have that $ub_c \leq \smashoperator[r]{\sum_{x. a_x>0}}a_x\cdot ub_x + \smashoperator[r]{\sum_{x.a_x<0}}a_x\cdot lb_x$ at a feasible point, while $lb_c \geq \smashoperator[r]{\sum_{x. a_x>0}}a_x\cdot lb_x + \smashoperator[r]{\sum_{x.a_x<0}}a_x\cdot ub_x$. Thus, 
tightening a bound of a variable or constraint can imply a tightening of another bound of a variable or constraint. 

The individual preprocessing steps are as follows; details can be found in \cite{MyTechRep}. 
%
\begin{enumerate} \itemsep0em
	\item Remove all empty inequalities and unused variables $x\in Y$.
%
	\item If $ub_x = lb_x$ for a variable $x\in X$, then substitute $x$ with $lb_x$ in all constraints in $S$. 
%
	\item If an inequality $a_x\cdot x \leq b$ belongs to $S$, then remove it from $S$ and update the appropriate bound: If $a_x>0$, update $ub_x$ to $\min\{ub_x,\frac{b}{a_x}\}$, otherwise update $lb_x$ to $\max\{lb_x,\frac{b}{a_x}\}$.
For the equality $a\cdot x = b$, both bounds are updated.
%
	\item If $\Neg_S(x)=\emptyset$ for an $x\in Y$ (implying that $x$ does not occur in an equality and has no (finite) lower bound) then remove $\Pos_S(x)$ from $S$. If $\Neg_S(x)$ only consist of the inequality defining the lower bound of $x$, $-x\leq -lb_x$, then substitute $x$ with $lb_x$ in all constraints in $S$. Similarly w.r.t. $\Pos_S(x)$. We notice that this is \emph{not} a normal preprocessing step, but corresponds to making a FME on $x$, which changes the feasible area of $S$. 
%
	\item Assume $c$ is a constraint with right-hand-side $b$. If ${ub}_c \leq b$, it is redundant and removed if it is an inequality and otherwise (if it is an equality), then necessarily ${ub}_c = b$, and $c$ is removed from $S$ while all positive (respectively negative) variables in $c$ are replaced with their upper (respectively lower) bound.
If $\mi{lb}_c \geq b$, then necessarily $\mi{lb}_c = b$, and $c$ is removed from $S$ while all positive (respectively negative) variables in $c$ are replaced with their lower (respectively upper) bound.
%
	\item For any variable $x$ used by $c$, if $a_x > 0$ and $\mi{lb}_{c,x}<+\infty$, 
	where $\mi{lb}_{c,x} = \smashoperator[r]{\sum_{x'.a_{x'}>0, x'\neq x}}a_{x'}\cdot lb_{x'} + \smashoperator[r]{\sum_{x'.a_{x'}<0}}a_{x'}\cdot ub_{x'}$, 
	then $ub_x$ is updated to $\min\{ub_x, \frac{b-\mathit{lb}_{c,x}}{a_x}\}$. 
	Similarly, if $a_x < 0$ and $\mi{lb}_{c,x}<+\infty$, then $lb_x$ is updated to $\max\{lb_x, \frac{b-\mathit{lb}_{c,x}}{a_x}\}$.	
	Updating bounds for constraints with only one variable (step 3) is a special case of this.
\setcounter{counterName}{\value{enumi}}
\end{enumerate}	
%
Comparing two inequalities syntactically, we can in some cases detect redundancy as described below. All pairs of inequalites are then compared in an efficient manner.
In the following, let $c$ be an constraint in $S$ with left-hand-side $\ve{a}\cdot \ve{x}$ and right-hand-side $b$, and let $c'$ be another constraint with left-hand-side $\ve{a}'\cdot \ve{x}$ and right-hand-side $b'$. 
\begin{enumerate} \itemsep0em
\setcounter{enumi}{\value{counterName}}
\item 
If $\ve{a}=\sigma\cdot \ve{a}'$, then $c$ is \emph{linearly dependent} (\cite{lassez93}). $c$ is hence redundant if
it is an \underline{in}equality and $\sigma\cdot b'\leq b$ and either: $\sigma\geq 0$; or $\sigma<0$ and and $c'$ is an equality. $c$ is also redundant if both $c$ and $c'$ are equalities and $\sigma\cdot b'\leq b$. 
It is enough to check if this holds for $\sigma = \frac{c_x}{c'_x}$ for the first variable $x$ used by $c'$.
\item
If all variables used in $c$ and $c'$ are non-negative, and there exists a $\sigma\geq 0$ such that $a_x \leq \sigma \cdot a'_x$ for all $x$, while $\sigma\cdot b' \leq b$, then $c$ is \emph{less strict} than $c'$; it is redundant and hence removed. If $c$ is an equality, so must $c'$ be, and hence $c'$ is turned into an equality. 
This is only required to be tested for specific value of $\sigma$ that depends on whether $c'$ has a variable with a negative coefficient or not; see \cite{MyTechRep}.
\end{enumerate} 